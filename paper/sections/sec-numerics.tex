% !TEX root = ../EntropicJKO.tex
\section{Numerical Results}
\label{sec-numerics}

We now illustrate the usefulness and versatility of our approach to compute approximate solutions to various non-linear diffusion processes. The videos showing the time evolutions displayed in the figures below are available online\footnote{\url{https://github.com/gpeyre/2015-SIIMS-wasserstein-jko/tree/master/videos}}.

%%%%%%%%%%%%%%%%%%%%%%%%%%%%%%%%%%%%%%%%%%%%%%%
\subsection{Exact and Approximate Kernel Computation}
\label{subsec-kernel-comp}

As already highlighted in Section~\ref{subsec-dykstra-jko}, our method is efficient if one can compute in a fast way the multiplication $\xi p$ between the Gibbs kernel $\xi=e^{-c/\ga}$ and a vector $p \in \RR^N$.  In the general case, this is intractable because this is a full matrix-vector multiplication. Even if $\xi$ usually has an exponential decay away from the diagonal, precisely capturing this decay is important to achieve transportation of mass effects. In particular, truncating the kernel to obtain a sparse matrix is prohibited. 

%%%
\paragraph{Translation invariant metrics}

The simplest setting is when the sampling points $(x_i)_{i=1}^N$ (as defined in Section~\ref{sec-entropic-jko}) correspond to an uniform grid discretization of a translation invariant distance, i.e. $c_{i,j} = D(x_i-x_j)^\al$ for  some function $D : \RR^d \rightarrow \RR$. In this case, $\xi$ is simply a discrete convolution against the kernel $D(\cdot)^\al$ sampled on an uniform grid. For instance, when $D(\cdot) = \norm{\cdot}$ and $\al=2$, $\xi$ is simply a convolution with a Gaussian kernel of width $\ga$. When using periodic or Neumann boundary conditions, it is thus possible to implement this convolution in $O(N\log(N))$ operations using Fast Fourier Transforms (FFT's). There also exists a flurry of linear time approximate convolutions, the most popular one being Deriche's factorization with IIR filters~\cite{deriche-1993}. We used this method to generate the top rows of Figure~\ref{fig-examples} and~\ref{fig-pairwise-attraction}. The other figures require a more advanced treatment because the kernel is not translation invariant. We now detail this approach.

%%%
\paragraph{Riemannian metrics}

Unfortunately, many case of practical interest correspond to diffusions inside non-convex domains, or even on non-Euclidean domains, typically a Riemannian manifold $\Mm$ (possibly with a boundary). In this setting, the natural choice for the ground cost $c$ is to set $c_{i,j} = d_\Mm(x_i,x_j)^\al$, where $d_\Mm$ is the geodesic distance on the manifold. 
%
A major issue is that computing this matrix $c$ is costly, for instance it would take $O(N^2\log(N))$ using Fast-Marching technics~\cite{sethian-book}�on a grid or a triangulated mesh of $N$ points. Even storing this non-sparse matrix can be prohibitive, and applying it at each step of the Dykstra algorithm would require $O(N^2)$ operations. 
%
Inspired by the ``geodesic in heat'' method of~\cite{crane-2013}, it has recently been proposed by~\cite{ConvolutionalOT}�to speed up these computations by approximating the kernel $\xi$ by a Laplace or a heat kernel $\tilde\xi$ (depending on wether $\al=1$ or $\al=2$). This means that $c$ does not need to be pre-computed and stored, and that, as explained below, one can apply it on the fly at each iteration using a fast sparse linear solver. 
%
In the numerical tests, we have used this approximation. 

To this end, one only needs to have at its disposal a discrete approximation $\De_\Mm$ of the Laplacian operator on the manifold $\Mm$. This is easily achieved using axis-aligned finite differences for manifold discretized on a rectangular grid, and this is the case for Figures~\ref{fig-examples}. When considering a discretized manifold $\Mm$ which is a triangulated surface (as this is the case for Figure~\ref{fig-meshes}), one can use piecewise linear $P_1$ finite elements, and the operator $\De_\Mm$ is then the popular Laplacian with cotangent weights, see~\cite{botsch-2010}. 

Following~\cite{ConvolutionalOT}, the kernel $\xi$ is then approximated using $L \in \NN^*$ steps of implicit Euler time discretization of the resolution of the heat equation on $\Mm$ until time $\ga$, i.e.
\eql{\label{eq-heat-kernel}
	\xi \approx \tilde\xi \eqdef \pa{ \Id - \frac{\ga}{L} \Delta_\Mm }^{-L} 
}
where $(\cdot)^{-L}$ means that one iterates $L$ times matrix inversion. 

When $L=1$, and ignoring discretization errors, the result of Varadhan~\cite{varadhan-1967} shows that in the limit of small $\ga$, $\tilde\xi$ converges to the kernel $\xi$ obtained when using $\al=1$ (i.e. ``$W_1$'' optimal transport). 
%
Indeed, this formula state that $-\frac{1}{\ga} \log(\tilde\xi)$ converge uniformly when $\ga \rightarrow 0$ toward the geodesic distance on $\Mm$.
%
As $L$ increases, $\tilde\xi$ converges to the heat kernel, which can be shown, also using~\cite{varadhan-1967} to be consistent with the case $\al=2$ (i.e. ``$W_2$'' optimal transport). In the numerical tests, we have used $L=10$. 

Numerically, the computation of matrix/vector multiplications $\tilde\xi p$ appearing the Dykstra updates thus requires the resolution of $L$ linear systems. Since these matrix/vector multiplications are computed many times during the iterations, a considerable speed-up is achieved by first pre-computing a sparse Cholesky factorization of $\Id - \frac{\ga}{L} \Delta_\Mm$ and then solving the $L$ linear systems by back-substitution~\cite{Davis:2006}. Although there is no theoretical guarantee, we observed numerically that each step typically has a linear time complexity because the factorization is indeed highly sparse. We refer to~\cite{crane-2013} for an experimental analysis of this class of Laplacian solvers.  


%%%%%%%%%%%%%%%%%%%%%%%%%%%%%%%%%%%%%%%%%%%%%%%
\subsection{Numerical Settings}

In the following sections, we show simulations on 2-D domains, either on a flat square (possibly equipped with an anisotropic Riemannian metric) or on a surface embedded in $\RR^3$ (thus equipped with the induced Riemannian metric).
% 
The square domain is set to $[0,1]^2$ and we use an uniform grid of $N=100 \times 100$ points. 
% 
The surface used for the numerical tests is a triangulated mesh of $N = 20000$ vertices, fitting in the bounding box $[0,1]^3$. 

We consider only cost functions corresponding to squared geodesic distances, i.e. we use $\al=2$. 
%
When the cost function is $c_{i,j} = \norm{x_i-x_j}^2$, we implemented multiplications by $\xi$ with a fast Gaussian filtering as explained in Section~\ref{subsec-kernel-comp}.
%
For cost functions corresponding to geodesic distances $c_{i,j} = d_\Mm(x_i,x_j)^2$, we implemented the heat kernel approximation~\eqref{eq-heat-kernel} with $L=10$. 
%
We always consider Neumann boundary conditions, which correspond, for the case of translation invariant kernels, to convolutions with symmetric extensions across boundaries. 

We set the entropic smoothing parameter to $\ga = 1/N$, and the time step to $\tau = 1/\sqrt{N}$. 
%
The variable $t$ indicates the iteration number, as defined in~\eqref{eq-smooth-jko}. 


%%%%%%%%%%%%%%%%%%%%%%%%%%%%%%%%%%%%%%%%%%%%%%%
\subsection{Crowd Motion Model}

To model crowd motion, we follow~\cite{maury2010macroscopic}, where a congestion effect (not too many people can be at the same position) is enforced by imposing that the density $p$ of peoples follows a JKO flow with the functional $f$ defined as
\eql{\label{eq-dfn-congestion}
	\foralls p \in \RR^N, \quad
	f(p) \eqdef \iota_{[0,\kappa]^N}(p) + \dotp{w}{p}
}
where $\kappa \geq \normi{p_{t=0}}$ is the congestion parameter (the smaller, the more congestion)
and $w \in \RR^N$ is a potential (the mass should move along the gradient of $w$).

\newcommand{\myfigM}[3]{\includegraphics[width=.19\linewidth]{#1-kappa#3/#1-kappa#3-#2}}

\begin{figure}[h!]
	\centering
	\begin{tabular}{@{}c@{\hspace{1mm}}c@{\hspace{1mm}}c@{\hspace{1mm}}c@{\hspace{1mm}}c@{}}
		%%%%%%
		\myfigM{bump}{1}{10}&
		\myfigM{bump}{2}{10}&
		\myfigM{bump}{5}{10}&
		\myfigM{bump}{10}{10}&
		\myfigM{bump}{20}{10}\\
		\myfigM{bump}{1}{20}&
		\myfigM{bump}{2}{20}&
		\myfigM{bump}{5}{20}&
		\myfigM{bump}{10}{20}&
		\myfigM{bump}{20}{20}\\
		\myfigM{bump}{1}{40}&
		\myfigM{bump}{2}{40}&
		\myfigM{bump}{5}{40}&
		\myfigM{bump}{10}{40}&
		\myfigM{bump}{20}{40}\\
		$t=0$ & $t=10$ & $t=20$ & $t=30$ & $t=40$ 
	\end{tabular}
	\caption{% 
		Display of the influence of the congestion parameter $\kappa$ on the evolution (the driving potential $w$ is displayed on the upper-right corner of Figure~\ref{fig-examples}). 
		From top to bottom, the parameters are set to $\kappa/\normi{p_{t=0}} \in \{1, 2, 4\} \}$.
	}
   \label{fig-influ-kappa}
\end{figure}

For such a function, the KL proximal operator is easy to compute, as detailed in the following proposition.

\begin{prop}\label{prop-prox-congest}
	One has 
	\eql{\label{eq-dfn-congestion-prox}
		\foralls p \in \RR^N, \quad 
		\Prox_{\si f}^{\oKL}(p) = \min(p \odot e^{-\si w},\kappa)
	}	
	where the min should be understood components-wise. 
\end{prop}

\begin{proof}
	By separability of the functional, one only needs to prove the 1-D case.
	One first proves the formula when $w=0$.
	One then applies~\eqref{eq-prox-calculus-shift} in the case $M=1$.
\end{proof}	
	
	
Note that it is of course possible to consider a $\kappa$ that is spatially varying to model a non-homogeneous congestion effect. 

Figure~\ref{fig-influ-kappa} shows the influence of the congestion parameter $\kappa$. This figure is obtained for the quadratic Euclidean cost $c_{i,j}=\norm{x_i-x_j}^2$.  
%
Figure~\ref{fig-examples} shows various evolutions for different potentials $w$ (guiding the crowd to the exit) on a manifold $\Mm$ which is a sub-set of a square in $\RR^2$. This means that locally the Riemannian metric is Euclidean, but since the domain is non-convex, the transport is defined according to a geodesic distance $d_\Mm$ which is not the Euclidean distance. The discretization is achieved using the approximate heat kernel~\eqref{eq-heat-kernel}.

\newcommand{\myfig}[2]{\includegraphics[width=.16\linewidth]{#1-kappa10/#1-kappa10-#2}}
\newcommand{\myfigPot}[1]{\includegraphics[width=.16\linewidth]{potentials/#1-potential}}


\begin{figure}[h!]
	\centering
	\begin{tabular}{@{}c@{\hspace{1mm}}c@{\hspace{1mm}}c@{\hspace{1mm}}c@{\hspace{1mm}}c@{\hspace{1mm}}c@{}}
		%%%%%%
		\myfig{bump}{1}&
		\myfig{bump}{2}&
		\myfig{bump}{5}&
		\myfig{bump}{10}&
		\myfig{bump}{20}&
		\myfigPot{bump} \\
		\myfig{disk}{1}&
		\myfig{disk}{2}&
		\myfig{disk}{5}&
		\myfig{disk}{10}&
		\myfig{disk}{20}&
		\myfigPot{disk} \\
		\myfig{disk1}{1}&
		\myfig{disk1}{2}&
		\myfig{disk1}{5}&
		\myfig{disk1}{10}&
		\myfig{disk1}{20}&
		\myfigPot{disk1} \\
		\myfig{holes}{1}&
		\myfig{holes}{2}&
		\myfig{holes}{5}&
		\myfig{holes}{10}&
		\myfig{holes}{20}&
		\myfigPot{holes} \\
		\myfig{tworooms}{1}&
		\myfig{tworooms}{2}&
		\myfig{tworooms}{5}&
		\myfig{tworooms}{10}&
		\myfig{tworooms}{20}&
		\myfigPot{tworooms}\\
		$t=0$ & $t=10$ & $t=20$ & $t=30$ & $t=40$ & $\cos(w)$
	\end{tabular}
	\caption{% 
		Display of crowd evolution for $\kappa=\normi{p_{t=0}}$. 
		The rightmost column display equispaced level-sets of the driving potential $w$. 
	}
   \label{fig-examples}
\end{figure}

Lastly Figure~\ref{fig-meshes} shows the evolution on a triangulated mesh, which is also implemented using the same heat kernel, but this time on a 3-D triangulation using piecewise linear finite elements (hence a discrete Laplacian with cotangent weights~\cite{botsch-2010}). 


\renewcommand{\myfigPot}[1]{\includegraphics[width=.19\linewidth]{potentials/#1-potential}}

\begin{figure}[h!]
	\centering
	\begin{tabular}{@{}c@{\hspace{1mm}}c@{\hspace{1mm}}c@{\hspace{1mm}}c@{\hspace{1mm}}c@{}}
		%%%%%%
		\multirow{2}{*}[3em]{\myfigPot{moomoo}} &
		\myfigM{moomoo}{1}{10}&
		\myfigM{moomoo}{5}{10}&
		\myfigM{moomoo}{10}{10}&
		\myfigM{moomoo}{20}{10}\\
		&
		\myfigM{moomoo}{1}{60}&
		\myfigM{moomoo}{5}{60}&
		\myfigM{moomoo}{10}{60}&
		\myfigM{moomoo}{20}{60} \\
		$\cos(w)$ & $t=0$ & $t=15$ & $t=30$ & $t=45$ 
	\end{tabular}
	\caption{% 
		Display of the evolution $p_t$ on a triangulated surface. 
		From top to bottom, the congestion parameter is set to $\kappa/\normi{p_{t=0}} \in \{1, 6\}$. 
	}
   \label{fig-meshes}
\end{figure}


%%%%%%%%%%%%%%%%%%%%%%%%%%%%%%%%%%%%%%%%%%%%%%%
\subsection{Anisotropic Diffusion Kernels}
\label{sec-anisotropic}

We consider the crowd motion functional~\eqref{eq-dfn-congestion} over measures defined on $\Mm = \RR^2$ now equipped with a Riemannian manifold structure defined by some tensor field $T(x) \in \RR^{d \times d}$ of symmetric positive matrices. We use the heat kernel approximation detailed in Section~\ref{subsec-kernel-comp}. The kernel~\eqref{eq-heat-kernel} thus corresponds to a discretization of an anisotropic diffusion, which are routinely used to perform image restoration~\cite{WeickertBook}. As the anisotropy (i.e. the maximum ratio between the maximum and minium eigenvalues) of the tensors increases, the corresponding linear PDE becomes ill-posed, and traditional discretizations using finite differences are inconsistent, leading to unacceptable artifacts. We thus use the adaptive anisotropic stencils recently proposed in~\cite{FehrenbachMirebeau} to define the sparse Laplacian matrix discretizing the manifold Laplacian $\Delta_\Mm u(x) = \text{div}( T(x) \nabla u(x) )$. This discrete Laplacian is able to cope with highly anisotropic tensor fields. This is illustrated in Figure~\ref{fig-anisotropic}, which shows the impact of the anisotropy on the trajectory of the mass. The potential $w$ creates an horizontal movement of the mass, but the circular shape of the tensor orientations forces the mass to rather follow a curved trajectory. Ultimately, mass accumulates on the left side and the congestion effect appears.


\renewcommand{\myfig}[2]{\includegraphics[width=.16\linewidth]{#1-kappa10/#1-kappa10-#2}}
\renewcommand{\myfigPot}[1]{\includegraphics[width=.16\linewidth]{potentials/#1-potential}}

\begin{figure}[h!]
	\centering
	\begin{tabular}{@{}c@{\hspace{1mm}}c@{\hspace{1mm}}c@{\hspace{1mm}}c@{\hspace{1mm}}c@{\hspace{1mm}}c@{}}
		%%%%%%
		\myfig{aniso1}{1}&
		\myfig{aniso1}{5}&
		\myfig{aniso1}{10}&
		\myfig{aniso1}{15}&
		\myfig{aniso1}{20} &
		\myfigPot{aniso1} \\
		\myfig{aniso3}{1}&
		\myfig{aniso3}{5}&
		\myfig{aniso3}{10}&
		\myfig{aniso3}{15}&
		\myfig{aniso3}{20} & 
		\myfigPot{aniso3} \\
		\myfig{aniso4}{1}&
		\myfig{aniso4}{5}&
		\myfig{aniso4}{10}&
		\myfig{aniso4}{15}&
		\myfig{aniso4}{20} &
		\myfigPot{aniso4} \\
		$t=0$ & $t=15$ & $t=30$ & $t=45$ & $t=60$ & $\cos(w), T$ 
	\end{tabular}
	\caption{% 
		Display of the evolution $p_t$ using anisotropic diffusion kernels.
		%
		The right column displays in the background the level-sets of $w$ and the tensor field $T(x)$
		displayed as red ellipsoids. 
		An ellipsoid at point $x$ is oriented along the principal axis of $T(x)$, and the length/width ratio is proportional to the anisotropy of $T(x)$. The metric thus favors mass movements along the direction of the ellipsoids.
		%
		The anisotropy (ratio between maximum and minimum eigenvalues of $T(x)$)
		is set respectively from top to bottom to $\{2, 10, 30\}$ in each of the successive rows.
	}
   \label{fig-anisotropic}
\end{figure}

%%%%%%%%%%%%%%%%%%%%%%%%%%%%%%%%%%%%%%%%%%%%%%% 
\subsection{Non-linear Diffusions} % REMOVED

%%%%%%%%%%%%%%%%%%%%%%%%%%%%%%%%%%%%%%%%%%%%%%%
\subsection{Non-linear Diffusions}

To model non-linear diffusion equations, we consider (possibly space-varying) generalized entropies
\eql{\label{eq-defn-gen-entropies}
	f(p) \eqdef \sum_i b_i e_{m_i}(p_i)
	\qwhereq
	\foralls m \geq 1, 
	e_m(s) \eqdef \choice{
		s (\log(s)-1) \qifq m=1, \\
		s \frac{s^{m-1}-m}{m-1}	\qifq m>1.
	}
}
Here $(b_i)_{i=1}^N$ is a set of weights $b_i \geq 0$ that enable a specially varying diffusion strength, while $(m_i)_{i=1}^N$ is a set of exponents that enable to make the evolution more non-linear at certain locations. Note that the case $m=1$ corresponds to minus the entropy defined in~\eqref{eq-entropy-defn}. 

In the case of constant weights $b$ and exponents $m$, the gradient flows of these functionals lead to non-linear diffusions of the form $\partial_t p = b \Delta p^m$. The case $m=1$ is the usual linear heat diffusion, as considered in the initial work of~\cite{jordan1998variational}. The case $m=2$ is the so-called porous medium equation~\cite{otto2001geometry}, where diffusion is slower in areas where the density $p$ is small. In particular, solutions might have a compact support that evolves in time, on contrary to the linear heat diffusion where mass can travel with infinite speed.  

The following proposition, details how to compute the proximal operator of $h$.

\begin{prop}
The proximal operator of $f$ satisfies 
\eq{
	\Prox_{\si f}^{\oKL}(r) = (\Prox_{\si b_i e_{m_i}}^{\oKL}(r_i) )_{i=1}^N.
}
For $m=1$, the proximal operator of $e_1$ reads
\eql{\label{eq-prox-entropy}
	\foralls s>0, \quad \Prox_{\si e_1}^{\oKL}(s) = s^{\frac{1}{1+\si}}.
}
If $m \neq 1$, then for any $s>0$, $\ProxKL_{\si e_m}(s) = \psi$ is the unique positive root of the equation
\eql{\label{eq-prox-psi}
	\log(\psi) + m \si \frac{\psi^{m-1} - 1}{m-1}  = \log(s)
}
\end{prop}

\begin{proof}
	The proof follows from writing the first order optimality condition of~\eqref{eq-defn-proxKL}, which are exactly~\eqref{eq-prox-psi}. For $m=1$, this equation can be solved in closed form. 
\end{proof}

In the numerical applications, we compute $\ProxKL_{\si e_m}$ by using a few steps of Newton iterations to solve~\eqref{eq-prox-psi}, which can be parallelized over all the grid's locations. Figure~\ref{fig-Psi} shows examples of the energy $e_m$ and the corresponding proximal maps $\ProxKL_{\si e_m}$. They act as pointwise non-linear thresholding operators that are applied iteratively on the probability distribution being computed. In some sense, the congestion term~\eqref{eq-dfn-congestion}�and the corresponding proximal operator~\eqref{eq-dfn-congestion-prox}�can be understood as a limit of this model as $m \rightarrow +\infty$.




\newcommand{\myfigProx}[1]{\includegraphics[width=.32\linewidth]{prox/#1}}

\begin{figure}[h!]
	\centering
	\begin{tabular}{@{}c@{\hspace{1mm}}c@{\hspace{1mm}}c@{}} % {@{}c@{\hspace{1mm}}c@{\hspace{1mm}}c@{}}
		\myfigProx{entropies3} &
		\myfigProx{prox_sigma1} &
		% \myfigProx{prox_sigma2} &
		\myfigProx{prox_sigma3} \\ 
		$e_m$ & $\ProxKL_{\si e_m}, \sigma=1$ & $\ProxKL_{\si e_m}, \sigma=3$ \\
	\end{tabular}
	\caption{% 
		Display of the graphs of functions $e_m$ and $\ProxKL_{\si e_m}$ for some values of $(\si,m)$.
	}
   \label{fig-Psi}
\end{figure}

We first consider an homogeneous 1-D setting with $b_i=b$ and $m_i = m$. This corresponds to the discretization of the porous medium equation $\partial_t p = b \Delta p^m$. Following~\cite{Westdickenberg2010} we set $b=\frac{(m-1)^2}{4m}$. There exists a family of explicit solutions, the so-called Barenblatt profiles, see~\cite{Westdickenberg2010} for instance, given by the expressions, for $t>-t_0$, 
\eql{\label{eq-barenblatt}
	(t+t_0)^{-\al}  \pa{ C^2 - k (t+t_0)^{-2\al}  x^2 }_{+}^{ \frac{1}{m-1} }
	\qwhereq
	\choice{
	\al = \frac{1}{m+1}, \\
	k = \frac{m-1}{2 m (m+1)}.
	}
}
where $t_0$ is a time shift and $C>0$ is a constant that controls the total mass of the solution. 

Figure~\eqref{fig-barenblatt} shows a comparison between the ground trust solution~\eqref{eq-barenblatt} and the approximation computed by the entropic gradient flow for $m = 4$, $t_0 = 1$, $C = 1/20$, on a grid of $N=2048$ points. The extra-smoothing added by the entropic scheme is clearly visible and it increases roughly linearly with time, so that the support of the solution is less and less compact. This is the main weakness of this numerical scheme, so that more conservative scheme such as~\cite{Westdickenberg2010}�should be considered, at least for this homogeneous 1-D setting.


\newcommand{\myfigBaren}[1]{\includegraphics[width=.48\linewidth]{barenblatt/barenblatt-t#1}}

\begin{figure}[h!]
	\centering
	\begin{tabular}{@{}c@{\hspace{1mm}}c@{}}
		%%%%%%
		\myfigBaren{0} & \myfigBaren{100}  \\
		$t=0$ & $t=100$ \\
		\myfigBaren{10} & \myfigBaren{200}  \\
		$t=10$ & $t=200$ \\
		\myfigBaren{50} & \myfigBaren{1000}  \\
		$t=50$ & $t=1000$ 
	\end{tabular}
	\caption{% 
		Comparison of the Barenblatt profile (dashed curves) and the approximated solution (plain curves) for $m=4$.  
	}
   \label{fig-barenblatt}
\end{figure}




Figure~\eqref{fig-porous} shows illustration of the models in the case where either $b$ or $m$ is varying, thus producing a spatially varying flow. The initial distribution $p_{t=0}$ is computed as a white noise realization, where the pixels are independently and identically drawn according to a uniform distribution on $[0,1]$ (and then $p$ is normalized to unit mass). 

\newcommand{\myfigPor}[2]{\includegraphics[width=.19\linewidth]{rand-varying-#1/rand-varying-#1-#2}}

\begin{figure}[h!]
	\centering
	\begin{tabular}{@{}c@{\hspace{1mm}}c@{\hspace{1mm}}c@{\hspace{1mm}}c@{\hspace{1mm}}c@{}}
		%%%%%%
		\myfigPor{e}{1}&
		\myfigPor{e}{5}&
		\myfigPor{e}{10}&
		\myfigPor{e}{15}&
		\myfigPor{e}{20}\\
		\myfigPor{m}{1}&
		\myfigPor{m}{5}&
		\myfigPor{m}{10}&
		\myfigPor{m}{15}&
		\myfigPor{m}{20}\\
		$t=0$ & $t=10$ & $t=20$ & $t=30$ & $t=40$ 
	\end{tabular}
	\caption{% 
		Non-linear diffusion by gradient flow on the generalized entropies~\eqref{eq-defn-gen-entropies}. 
		Top row: fixed $m_i=1.4$ and varying weights $b_i \in [1,20]$ (1 in the boundary, 20 in the center).
		Bottom row: fixed $b_i=1$ and varying exponents $m_i \in [1,2]$ (1 in the boundary, $2$ in the center).
	}
   \label{fig-porous}
\end{figure}


%%%%%%%%%%%%%%%%%%%%%%%%%%%%%%%%%%%%%%%%%%%%%%%
\subsection{Non-convex Functionals}

It is formally possible to apply Dykstra's algorithm detailed in Section~\ref{sec-kl-jko-algo} to a non-convex function $f$, if one is able to compute in closed form the proximal operator~\eqref{eq-defn-proxKL} (which then might be a multi-valued map). Of course there is no hope for the resulting non-convex Dykstra's algorithm to converge in general to the global minimizer of the non-convex optimization~\eqref{eq-defn-KL-prbm}. Even worse, to the best of our knowledge, there is currently no proof that the non-convex Dykstra's algorithm converges to a stationary point of the energy, even in the case of an Euclidean divergence. However, we found that applying Dykstra's algorithm to non-convex functions works remarkably well in practice. Note that the closely related Douglas-Rachford (DR) algorithm is known to converge in some particular non-convex cases~\cite{ArtachoNonCvx}. DR is known to perform very well on several non-convex optimization problems such as phase retrieval~\cite{BauschkeNonCvx}. 

To test this non-convex setting, we replace the congestion box constraint~\eqref{eq-dfn-congestion} by the non-convex function
\eql{\label{eq-dfn-congestion-noncvx}
	f(p) \eqdef \iota_{\{0,\kappa\}^N} + \dotp{w}{p}.
}
This function enforces that the thought after solution is binary, so that each value $p_i$ is in $\{0,\kappa\}$.
The proximal operator of this non-convex function can be computed explicitly using
\eq{
	\foralls i \in \{1,\ldots,N\}, \quad
	\Prox_{\si \iota_{\{0,\kappa\}^N}}^{\oKL}(p)_i = \choice{
		0 \qifq p_i < \kappa/e, \\
		\{0,\kappa\} \qifq p_i = \kappa/e, \\
		\kappa \qifq p_i > \kappa/e, 
	}
}
where $e=\exp(1)$. 
%
Note that $\Prox_{\si \iota_{\{0,\kappa\}^N}}^{\oKL}(p)_i$ is multi-valued at $p_i = \kappa/e$, and numerically one needs to chose one of the two values. 
%
Figure~\ref{fig-nonconvex} shows a comparison of the evolutions obtained with the convex and non-convex functionals. The non-convex one suffers from binary noise artefacts, which could be partly due to the non-convexity, but also to the amplification of discretization errors by the  proximal mapping which is not Lipschitz continuous. 

\renewcommand{\myfig}[2]{\includegraphics[width=.19\linewidth]{#1-kappa10/#1-kappa10-#2}}

\begin{figure}[h!]
	\centering
	\begin{tabular}{@{}c@{\hspace{1mm}}c@{\hspace{1mm}}c@{\hspace{1mm}}c@{\hspace{1mm}}c@{}}
		%%%%%%
		\myfig{tworooms}{1}&
		\myfig{tworooms}{2}&
		\myfig{tworooms}{5}&
		\myfig{tworooms}{10}&
		\myfig{tworooms}{20}\\
		\myfig{tworooms-noncvx}{1}&
		\myfig{tworooms-noncvx}{2}&
		\myfig{tworooms-noncvx}{5}&
		\myfig{tworooms-noncvx}{10}&
		\myfig{tworooms-noncvx}{20}\\
		$t=0$ & $t=10$ & $t=20$ & $t=30$ & $t=40$
	\end{tabular}
	\caption{% 
		Display of crowd evolution for $\kappa=\normi{p_{t=0}}$. 
		Top row: convex function~\eqref{eq-dfn-congestion}.
		Bottom row: non-convex function~\eqref{eq-dfn-congestion-noncvx}.
	}
   \label{fig-nonconvex}
\end{figure}



