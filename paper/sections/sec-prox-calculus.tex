% !TEX root = ../EntropicJKO.tex
\section{KL Proximal Calculus}
\label{sec-kl-calculus}

The following proposition details some useful property of the $\oKL$ proximal operator~\eqref{eq-def-kl-prox-operator}. This enables a powerful ``proximal calculus'' by combining these rules, which eases and simplifies the implementation of the algorithms. Note that we also consider generalized KL divergence over sets $(p_1,\ldots,p_M)$ of $M$ densities according to some weight $\la \in \RR_+^M$
\eql{\label{eq-defn-okl-lambda}
	\foralls (p_m)_m, (q_m)_m, \quad
	\oKL_\la( (p_m)_m | (q_m)_m ) \eqdef \sum_{m=1}^M \la_m \oKL(p_m|q_m).
}

% \newcommand{\proxrule}[1]{\noindent\textbf{(#1)}~}
\newcommand{\proxrule}[1]{}

\begin{proposition}
		\proxrule{Singleton constraint}�For $f(p_1,\ldots,p_M) \eqdef \iota_{\{(q_1,\ldots,q_M)\}}(p_1,\ldots,p_M)$, one has
			\eql{\label{eq-prox-calculus-indic}
				\Prox_{f}^{\oKL_\la}(p_1,\ldots,p_M)=(q_1,\ldots,q_M).
			}
		\proxrule{Linear shift} For $f(p_1,\ldots,p_M) \eqdef h(p_1,\ldots,p_M)+\sum_{i=1}^M \dotp{w_i}{p_i}$, one has
			\eql{\label{eq-prox-calculus-shift}
				\Prox_{f}^{\oKL_\la}(p_1,\ldots,p_M) = 
				\Prox_{h}^{\oKL_\la}(p_1 \odot e^{-w_1/\la_1}, \ldots, p_M \odot e^{-w_M/\la_M}).
			}
		\proxrule{Equality constraint} For $f(p_1,\ldots,p_M) \eqdef \iota_\Dd(p_1,\ldots,p_M)+h(p_1,\ldots,p_M)$ where
			\eq{
				\Dd \eqdef \enscond{(p_1,\ldots,p_M)}{p_1=\ldots=p_M},
			} 
			one has
			\eql{\label{eq-prox-calculus-equal}
				\Prox_{f}^{\oKL_\la}(p_1,\ldots,p_M)= (p,\ldots,p)
				\!\qwhereq\!
				p = \Prox_{\frac{1}{\sum_i \la_i} \tilde h}^{\KL}\pa{ p_1^{\tilde\la_1} \odot \ldots \odot p_M^{\tilde\la_M} }, 	
			}
			where we denoted $\tilde\la_i \eqdef \la_i/\sum_{j} \la_j$ and $\tilde h(p)=h(p,\ldots,p)$. \\
		\proxrule{Composition with sum} For  $f(p_1,\ldots,p_M) \eqdef h(p_1+\ldots+p_M)$ and 
			$\la \eqdef (1,\ldots,1)$, one has 
			\eql{\label{eq-prox-calculus-sum}
				\Prox_{f}^{\oKL_\la}(p_1,\ldots,p_M)= \frac{\Prox_{h}^{\oKL}(p_1+\ldots+p_M)}{p_1+\ldots+p_M} (p_1,\ldots,p_M)
			}
		\proxrule{Lifting} We define $f(\pi_1,\ldots,\pi_M) \eqdef h(\pi_1\ones,\ldots,\pi_M\ones)$. 
			We denote 
			\eq{
				\foralls m \in \{1,\ldots,M\}, \quad p_m \eqdef \pi_m\ones
				\qandq
				(\tilde p_1,\ldots,\tilde p_M) \eqdef  \Prox_h^{\oKL_\la}( p_1,\ldots,p_M ).
			}
			One has
			\eql{\label{eq-prox-calculus-lifting}
				\Prox_{f}^{\KL_\la}(\pi_1,\ldots,\pi_M) =
				\pa{
					\diag\pa{ \frac{\tilde p_m}{p_m} }�\pi
				}_m
			}
		\proxrule{Lifting, transposed} We define $f(\pi_1,\ldots,\pi_M) \eqdef h(\pi_1^T\ones,\ldots,\pi_M^T\ones)$. 
			We denote 
			\eq{
				\foralls m \in \{1,\ldots,M\}, \quad p_m \eqdef \pi_m^T \ones
				\qandq
				(\tilde p_1,\ldots,\tilde p_M) \eqdef  \Prox_h^{\oKL_\la}( p_1,\ldots,p_M ).
			}
			One has
			\eql{\label{eq-prox-calculus-lifting-transp}
				\Prox_{f}^{\KL_\la}(\pi_1,\ldots,\pi_M) =
				\pa{
					\pi \diag\pa{ \frac{\tilde p_m}{p_m} }
				}_m
			}
\end{proposition}

\newcommand{\proxruleP}[1]{\noindent\textbf{Proof of \eqref{eq-prox-calculus-#1}. }~}
\begin{proof}
%	\begin{enumerate}
	%%%%%%%%%%%%%%%%%%%%%%%%%%%%%%%%%%%%%%%%%%	
		\proxruleP{indic} This is straightforward.  \\
	%%%%%%%%%%%%%%%%%%%%%%%%%%%%%%%%%%%%%%%%%%	
		\proxruleP{shift}  If follows from the relation
		\begin{align*}
			&\oKL_\la( (q_1,\ldots,q_M)|(p_1,\ldots,p_M) ) + \sum_{i=1}^M \dotp{w_i}{q_i} \\ =  
			&\oKL_\la( (q_1,\ldots,q_M)| (p_1 \odot e^{-w_1/\la_1}, \ldots, p_M \odot e^{-w_M/\la_M}) ).
		\end{align*}
	%%%%%%%%%%%%%%%%%%%%%%%%%%%%%%%%%%%%%%%%%%	
		\proxruleP{equal}  
		We denote $(q_m)_m \eqdef \Prox^{\oKL_\la}_{\psi_1}( (p_m)_m )$, so that $q=q_1=\ldots=q_M$ solves
	\eq{
		\umin{q} \sum_m \la_m\oKL(q|p_m) + \tilde h(q).
	}
	The result follow from the relation
	\eq{
		\sum_m \la_m \oKL(q|p_m) = 
		\textstyle \pa{\sum_m \la_m} \oKL\pa{q| p_1^{\tilde\la_1} \odot \ldots \odot  p_M^{\tilde\la_M} 
		}.
	}
	%%%%%%%%%%%%%%%%%%%%%%%%%%%%%%%%%%%%%%%%%%	
		\proxruleP{sum}  Denoting $(q_m)_m = \Prox_{f}^{\oKL_\la}( (p_m)_m )$, the first order optimality condition for $\Prox_{f}^{\oKL_\la}$ reads 
	\eq{
		\foralls m \in \{1,\ldots,M\}, \quad
			\log\pa{\frac{q_m}{p_m}} + u = 0 
	}
	where $u \in \partial h(p_1+\ldots+p_M)$. Respectively summing and subtracting these equations lead to
	\eq{
		q_1+\ldots+q_M = \Prox_{h}(p_1+\ldots+p_M)
		\qandq
		\frac{q_1}{p_1} = \ldots = \frac{q_m}{p_m}.
	}
	Solving for $(q_1,\ldots,q_m)$ in these equations leads to the desired solution. \\
	%%%%%%%%%%%%%%%%%%%%%%%%%%%%%%%%%%%%%%%%%%	
	\proxruleP{lifting} The first order condition for $\tilde\pi$ being a solution of~\eqref{eq-defn-proxKL}�states the existence of $(z_m)_m \in \partial f(\tilde p_1,\ldots,\tilde p_M)$ where $\tilde p_m = \tilde\pi_m \ones$ such that 
	\eq{
		\la_m \log\pa{ \frac{\tilde\pi_m}{\pi_m} } + z_m \ones^T = 0
		\;\Rightarrow\;
		\tilde\pi_m = \diag(e^{-z_m/\la_m }) \pi_m
		\;\Rightarrow\;
		\tilde p_m = \diag(e^{-z_m/\la_m }) p_m, 
	}	 
	which corresponds to the first order condition for $(\tilde p_m)_m$ being a solution of~\eqref{eq-def-kl-prox-operator} for the function $h$, i.e.
	\eq{
		(\tilde p_m)_m =  \Prox_{h}^{\oKL}( (p_m)_m ).
	} 
	Finally, one obtains
	\eq{
		\tilde\pi_m = \diag(e^{-z_m/\la_m}) \pi_m = \diag\pa{ \frac{\tilde p_m}{p_m} } \pi_m
	}	
	and hence the desired result. 	\\		
	%%%%%%%%%%%%%%%%%%%%%%%%%%%%%%%%%%%%%%%%%%	
		\proxruleP{lifting-transp} It is obtained by transposing formula~\eqref{eq-prox-calculus-lifting}. 
%	\end{enumerate}	
\end{proof}
